\usepackage{graphicx,color,url,booktabs,comment}  %cite
\usepackage{authblk}
\usepackage{amsmath, amssymb}
\usepackage{mathrsfs}
\usepackage{qcircuit}
\usepackage{braket}
\usepackage{algorithm}
\usepackage{algpseudocode}
\usepackage{cite,xcolor,float}
\usepackage[utf8]{inputenc}
\usepackage{tabularx}
\usepackage[pdfstartview=FitH,colorlinks,linkcolor=blue,filecolor=blue,citecolor=blue,urlcolor=blue]{hyperref}
\usepackage{fixltx2e}
\MakeRobust{\Call}
\usepackage[numbers]{natbib} % \citet{foo} -> Foo et al. [5]
\usepackage{xspace}
\usepackage{cleveref} %% for referencing use \cref
\usepackage{mdframed}

\usepackage{amsfonts}

\usepackage{unicode-math}

\setmathfont{lmmath-regular.otf}[
Extension={.otf},
Path=./fonts/,
Scale=1]

\setmathfont{STIX2Math}[
Extension={.otf},
Path=./fonts/,
range=scr/{latin},
Scale=1]

\let\mathbb\relax % remove the definition by unicode-math
\DeclareMathAlphabet{\mathbb}{U}{msb}{m}{n}

\let\proof\relax
\let\endproof\relax
\usepackage{amsthm}
\urlstyle{sf}
%\usepackage[margin=1in]{geometry}
% \usepackage{fancyhdr}
%\usepackage[colorlinks]{hyperref} %pagebackref

\newcommand{\nc}{\newcommand}
\nc{\rnc}{\renewcommand}
%
%\newcommand{\bra}[1]{\langle #1|}
%\newcommand{\ket}[1]{|#1\rangle}
\newcommand{\proj}[1]{|#1\rangle\langle #1|}
\nc{\dyad}[1]{\ket{#1}\!\bra{#1}}
% \newcommand{\braket}[2]{\langle #1|#2\rangle}
% \newcommand{\Bra}[1]{\left\langle #1\right|}
% \newcommand{\Ket}[1]{\left|#1\right\rangle}
\newcommand{\Proj}[1]{\left|#1\right\rangle\left\langle #1\right|}
% \newcommand{\Braket}[2]{\left\langle #1\middle|#2\right\rangle}
\nc{\vev}[1]{\langle#1\rangle}
\nc{\grad}{{\vec{\nabla}}}
\nc{\abs}[1]{\lvert#1\rvert}
\nc{\onreg}[2]{\ensuremath{{#1}^{\gray{#2}}}}
%\DeclareMathOperator{\abs}{abs}
\DeclareMathOperator{\Bin}{Bin}
\DeclareMathOperator{\conv}{conv}
\DeclareMathOperator{\eig}{eig}
\DeclareMathOperator{\id}{id}
\DeclareMathOperator{\Img}{Im}
\DeclareMathOperator{\Par}{Par}
\DeclareMathOperator{\poly}{poly}
\DeclareMathOperator{\polylog}{polylog}
%\DeclareMathOperator{\negl}{negl}
\DeclareMathOperator{\tr}{tr}
\DeclareMathOperator{\rank}{rank}
% \DeclareMathOperator{\sgn}{sgn}
\DeclareMathOperator{\Sep}{Sep}
\DeclareMathOperator{\SepSym}{SepSym}
\DeclareMathOperator{\Span}{span}
\DeclareMathOperator{\supp}{supp}
\DeclareMathOperator{\swap}{SWAP}
\DeclareMathOperator{\Sym}{Sym}
\DeclareMathOperator{\ProdSym}{ProdSym}
\DeclareMathOperator{\SEP}{SEP}
\DeclareMathOperator{\PPT}{PPT}
\DeclareMathOperator{\Wg}{Wg}
\DeclareMathOperator{\WMEM}{WMEM}
\DeclareMathOperator{\WOPT}{WOPT}

\DeclareMathOperator{\BPP}{\mathsf{BPP}}
\DeclareMathOperator{\QPIP}{\mathsf{QPIP}}
\DeclareMathOperator{\SampBQP}{\mathsf{SampBQP}}
\DeclareMathOperator{\BQP}{\mathsf{BQP}}
\DeclareMathOperator{\cnot}{\normalfont\textsc{cnot}}
\DeclareMathOperator{\DTIME}{\mathsf{DTIME}}
\DeclareMathOperator{\NTIME}{\mathsf{NTIME}}
\DeclareMathOperator{\MA}{\mathsf{MA}}
\DeclareMathOperator{\NP}{\mathsf{NP}}
\DeclareMathOperator{\NEXP}{\mathsf{NEXP}}
\DeclareMathOperator{\Ptime}{\mathsf{P}}
\DeclareMathOperator{\QMA}{\mathsf{QMA}}
\DeclareMathOperator{\QCMA}{\mathsf{QCMA}}
\DeclareMathOperator{\BellQMA}{\mathsf{BellQMA}}

% \newcommand{\be}{\begin{equation}}
% \newcommand{\ee}{\end{equation}}
% \newcommand{\bea}{\begin{eqnarray}}
% \newcommand{\eea}{\end{eqnarray}}
% \newcommand{\nn}{\nonumber}
% \newcommand{\bi}{\begin{itemize}}
% \newcommand{\ei}{\end{itemize}}
% \newcommand{\bn}{\begin{enumerate}}
% \newcommand{\en}{\end{enumerate}}
% \def\beas#1\eeas{\begin{eqnarray*}#1\end{eqnarray*}}
% \def\ba#1\ea{\begin{align}#1\end{align}}
% \nc{\bas}{\begin{aligned}}
% \nc{\eas}{\end{aligned}}
% \nc{\bpm}{\begin{pmatrix}}
% \nc{\epm}{\end{pmatrix}}

\def\keq{\approx_{\negl(\kappa)}}

\def\non{\nonumber}
\def\nn{\nonumber}
\def\eq#1{(\ref{eq:#1})}
\def\eqs#1#2{(\ref{eq:#1}) and (\ref{eq:#2})}
%\def\eq#1{Eq.~(\ref{eq:#1})}
%\def\eqs#1#2{Eqs.~(\ref{eq:#1}) and (\ref{eq:#2})}
\def\L{\left} 
\def\R{\right}
\def\ra{\rightarrow}
\def\ot{\otimes}
\def\ol{\overline}

\ifdefined\LLNCS
\newtheorem{thm}{Theorem}
\nc{\thmautorefname}{Theorem}
\newtheorem{rmk}[remark]{Remark}
\newtheorem{cor}[corollary]{Corollary}
\newtheorem{lem}[lemma]{Lemma}
\theoremstyle{definition}
\newtheorem{dfn}[definition]{Definition}
\else
\newtheorem{thm}{Theorem}
\nc{\thmautorefname}{Theorem}
\newtheorem{rmk}{Remark}
\newtheorem{cor}[thm]{Corollary}
\newtheorem{lem}[thm]{Lemma}
\theoremstyle{definition}
\newtheorem{dfn}{Definition}
\fi


\ifdefined\LLNCS
\else
% \newtheorem{thm}{Theorem}
% \newtheorem{rmk}{Remark}
\newtheorem*{thm*}{Theorem}
%\newtheorem{claim}[thm]{Claim}
% \newtheorem{cor}[thm]{Corollary}
% \newtheorem{lem}[thm]{Lemma}
%\newtheorem{prop}[thm]{Proposition}
% \newtheorem{dfn}{Definition}
%\newtheorem{proto}{Protocol}
%\newtheorem{con}[thm]{Conjecture}


%\makeatletter
%\newtheorem*{rep@theorem}{\rep@title}
%\newcommand{\newreptheorem}[2]{%
%\newenvironment{rep#1}[1]{%
% \def\rep@title{#2 \ref{##1} (restatement)}%
% \begin{rep@theorem}}%
% {\end{rep@theorem}}}
%\makeatother

\makeatletter
\newtheorem*{rep@theorem}{\rep@title}
\newcommand{\newreptheorem}[2]{%
\newenvironment{rep#1}[1]{%
 \def\rep@title{#2 \ref{##1}}%
 \begin{rep@theorem}}%
 {\end{rep@theorem}}}
\makeatother

\newreptheorem{thm}{Theorem}
\newreptheorem{lem}{Lemma}
\fi

\def\eps{\epsilon}
\def\va{{\vec{a}}}
\def\vb{{\vec{b}}}
\def\vn{{\vec{n}}}
\def\cvs{{\cdot\vec{\sigma}}}
\def\vx{{\vec{x}}}
\def\Usch{U_{\text{Sch}}}

\def\cA{\mathcal{A}}
\def\cB{\mathcal{B}}
\def\cC{\mathcal{C}}
\def\cD{\mathcal{D}}
\def\cE{\mathcal{E}}
\def\cF{\mathcal{F}}
\def\cH{\mathcal{H}}
\def\cK{\mathcal{K}}
\def\cI{{\cal I}}
\def\cL{{\cal L}}
\def\cM{{\cal M}}
\def\cN{\mathcal{N}}
\def\cO{{\cal O}}
\def\cP{\mathcal{P}}
\def\cQ{\mathcal{Q}}
\def\cR{\mathcal{R}}
\nc\scrr{{\mathscr{r}}}
\def\cS{{\mathcal{S}}}
\nc\scrs{{\mathscr{s}}}
\def\cT{{\cal T}}
\def\cU{\mathcal{U}}
\def\cW{{\cal W}}
\def\cV{\mathcal{V}}
\def\cX{{\cal X}}
\def\cY{{\cal Y}}
\def\Enc{\mathsf{Enc}}
\def\Dec{\mathsf{Dec}}

\def\bp{\mathbf{p}}
\def\bq{\mathbf{q}}
\def\bP{{\bf P}}
\def\bQ{{\bf Q}}
\def\gl{\mathfrak{gl}}

\def\bbC{\mathbb{C}}
% \DeclareMathOperator*{\E}{\mathbb{E}}
\DeclareMathOperator*{\bbE}{\mathbb{E}}
\def\bbF{\mathbb{F}}
\def\bbM{\mathbb{M}}
\def\bbN{\mathbb{N}}
\def\bbR{\mathbb{R}}
\def\bbZ{\mathbb{Z}}
\def\bbP{\mathbb{P}}
\def\bbV{\mathbb{V}}
\newcommand{\Real}{\textrm{Re}}

\def\real{{\mathsf{Real}}}
\def\ideal{{\mathsf{Ideal}}}
\def\acc{{\mathsf{Acc}}}
\def\rej{{\mathsf{Rej}}}
\def\SA{{\mathsf{SA}}}
\def\route{{\mathsf{AR}}}
\def\ART{{\mathsf{RT}}}

\def\benum{\begin{enumerate}}
\def\eenum{\end{enumerate}}
\def\bit{\begin{itemize}}
\def\eit{\end{itemize}}
\def\bdesc{\begin{description}}
\def\edesc{\end{description}}
\newcommand{\fig}[1]{Figure~\ref{fig:#1}}
\newcommand{\tab}[1]{Table~\ref{tab:#1}}
\newcommand{\secref}[1]{Section~\ref{sec:#1}}
\newcommand{\appref}[1]{Appendix~\ref{sec:#1}}
\newcommand{\thmref}[1]{Theorem~\ref{thm:#1}}
\newcommand{\propref}[1]{Proposition~\ref{prop:#1}}
\nc{\protoref}[1]{\hyperref[#1]{Protocol~\ref*{#1}}}
\newcommand{\defref}[1]{Definition~\ref{def:#1}}
\newcommand{\corref}[1]{Corollary~\ref{cor:#1}}
\newcommand{\conref}[1]{Conjecture~\ref{con:#1}}

\newcommand{\FIXME}[1]{{\color{red}FIXME: #1}}
\nc{\todo}[1]{\textcolor{red}{todo: #1}}



\newcommand{\boxdfn}[2]{
\begin{figure}[h]
\begin{center}
\noindent \framebox{
\begin{minipage}{0.8\textwidth}
\begin{dfn}[{\bf #1}]
\ \\ \\
#2
\end{dfn}
\end{minipage}
}
\end{center}
\end{figure}
}

\newcommand{\boxproto}[2]{
\begin{figure}[h]
\begin{center}
\noindent \framebox{
\begin{minipage}{0.8\textwidth}
\begin{proto}[{\bf #1}]
\ \\ \\
#2
\end{proto}
\end{minipage}
}
\end{center}
\end{figure}
}

\def\begsub#1#2\endsub{\begin{subequations}\label{eq:#1}\begin{align}#2\end{align}\end{subequations}}
\nc\qand{\qquad\text{and}\qquad}
\nc\mnb[1]{\medskip\noindent{\bf #1}}
\nc\mn{\medskip\noindent}

\renewcommand{\arraystretch}{1.5}
%\nc{\problem}[1]{\item\noindent {\bf #1}}

\setlength{\tabcolsep}{10pt}

%%%%%% Han-Hsuan's commands %%%%%%%%
\nc{\nl}{\nn \\ &=}  %new line
\nc{\nnl}{\nn \\ &}  %new new line
\nc{\fot}{\frac{1}{2}} %frac one two
\nc{\oo}[1]{\frac{1}{#1}} % one over
\newcommand{\ben}{\begin{enumerate}}
\newcommand{\een}{\end{enumerate}}
\nc{\mc}{\mathcal}
% \nc{\beq}{\begin{equation}}
% \nc{\eeq}{\end{equation}}
% \nc{\norm}[1]{\L\| #1 \R\|}

\nc{\onenorm}[1]{\L\| #1 \R\|_1} %one norm
%\nc{\span}{\ensuremath{\mathrm{span}}}

\DeclareMathOperator*{\argmax}{arg\,max}

%\nc{1}

\newcommand{\Knote}[1]{\textcolor{red}{\small {\textbf{(KM:} #1\textbf{) }}}}

\nc{\Ra}{\Rightarrow}
\nc{\zo}{\{0,1\}}	


% \newenvironment{protocol}[1][htb]{\floatname{algorithm}{Protocol}\begin{algorithm}[#1]}
%  {\end{algorithm}}
\newcounter{protocol}
\newcommand{\linefill}{\rule{\linewidth}{0.8pt}}

\newenvironment{protocol}[1]{\begingroup\setlength\parindent{0pt}\medskip\noindent\linefill\\
\refstepcounter{protocol}\textbf{Protocol \theprotocol} #1\\\noindent\linefill}
{\vspace{-\topsep}\noindent\linefill\endgroup}
% \newenvironment{protocol}[1]
%   {\par\addvspace{\topsep}
%   \noindent
%   \tabularx{\linewidth}{@{} X @{}}
%     \toprule\vspace*{-\baselineskip}
%     \refstepcounter{protocol}\textbf{Protocol \theprotocol} #1 \\
%     \midrule\vspace*{-\baselineskip}}
%   {\\
%   \bottomrule
%   \endtabularx
%   \par\addvspace{\topsep}}

\ifdefined\LLNCS
\newaliascnt{claiml}{theorem}
\newtheorem{claiml}[claiml]{Claim}
\aliascntresetthe{claiml}

\renewenvironment{claim}{\begin{claiml}}{\end{claiml}}

\crefname{claiml}{Claim}{Claims}
\else\fi


%%\crefname{environment name}{singular ref}{plural ref}
\crefname{lemma}{Lemma}{Lemmata}
\crefname{figure}{Figure}{Figures}
\crefname{corollary}{Corollary}{Corollaries}
\crefname{proposition}{Proposition}{Propositions}
\crefname{conjecture}{Conjecture}{Conjectures}
\crefname{definition}{Definition}{Definitions}
\crefname{remark}{Remark}{Remarks}
\crefname{example}{Example}{Examples}
\crefname{algorithm}{Algorithm}{Algorithms}
\crefname{protocol}{Protocol}{Protocols}
%\Crefname{protocol}{Protocol}{Protocols}
%\crefformat{algorithm}{Algorithm~\textup{#2#1#3}}
\Crefformat{algorithm}{\textup{Algorithm~#2#1#3}}
%\Crefformat{protocol}{\textup{Protocol~#2#1#3}}

\renewcommand{\cref}{\Cref} %make all reference start with uppercase
\newcommand{\mcitet}[1]{\citet*{#1}}

% Proof of environments
\ifdefined\LLNCS
\newenvironment{proofof}[1]{\begin{proof}[of~#1]}{\end{proof}}
\newenvironment{proofsketchof}[1]{\begin{proofsketch}[of~#1]}{\end{proofsketch}}
\else
\newenvironment{proofof}[1]{\begin{proof}[Proof of~#1]}{\end{proof}}
\newenvironment{proofsketchof}[1]{\begin{proofsketch}[Proofsketch of~#1]}{\end{proofsketch}}
\fi

%%%%%%%%%%%%%%%%%%%%%%%%%%%%%%%%%%
% General commands.
%%%%%%%%%%%%%%%%%%%%%%%%%%%%%%%%%%
\newcommand{\sgn}{\operatorname{sgn}}
\DeclareMathOperator*{\E}{\mathbb{E}}
\DeclareMathOperator*{\spn}{\operatorname{span}}
\newcommand{\norm}[1]{\left\lVert#1\right\rVert}
\newcommand{\algorithmautorefname}{Algorithm}
\newcommand{\SD}{\operatorname{\Delta}}
\newcommand{\negl}{\operatorname{neg}}
% \newcommand{\ra}{\rightarrow}
\newcommand{\Tr}{\mathrm{Tr}}


%Acronyms
\newcommand{\wrt}{{with respect to \ }}
\newcommand{\ie}{{i.e.,\ }}
\newcommand{\eg}{{e.g.,\ }}
\newcommand{\swia}{security-with-identifiable-abort\xspace}
\newcommand{\wlg}{{without loss of generality\ }}
\newcommand{\cf}{{cf.,\ }}
\newcommand{\etal}{{et~al.\ }}
\newcommand{\aka}{{also known as,\ }}
\newcommand{\resp}{{resp.,\ }}

%MPC related commands
\newcommand{\parties}{\mathcal{P}}
\newcommand{\numparties}{n}
\newcommand{\Pc}{\operatorname{P}}
\newcommand{\I}{{\mathcal{I}}}
\newcommand{\adv}{{\mathcal{A}}}
\newcommand{\Out}{\operatorname{Out}}
\newcommand{\View}{\operatorname{View}}
\newcommand{\REAL}{\operatorname{REAL}}
\newcommand{\IDEAL}{\operatorname{IDEAL}}
\newcommand{\MS}{\operatorname{MS}}
\newcommand{\Sim}{\mathsf{S}}
\newcommand{\MPC}{\cMPC}
\newcommand{\cMPC}{\mathsf{cMPC}}
\newcommand{\trustp}{\mathsf{T}}
\newcommand{\abort}{\normalfont{\texttt{abort}}}
\newcommand{\continue}{\normalfont{\texttt{continue}}}
\newcommand{\success}{\normalfont{\texttt{success}}}
\newcommand{\aux}{{\mathsf{aux}}}
\newcommand{\func}{\mathcal{F}}

%Parentheses commands
\newcommand{\myvec}[1]{\normalfont{\textbf{#1}}}
\newcommand{\vect}[1]{\left(#1\right)}
\newcommand{\of}[1]{\left(#1\right)}
\renewcommand{\set}[1]{\left\{#1\right\}}
\newcommand{\ketbra}[1]{\dyad{#1}}

%General fonts
\newcommand{\st}{\mathsf{st}}
\newcommand{\discard}{\mathsf{discard}}
\newcommand{\accuse}{\mathsf{accuse}}
\newcommand{\intext}{\operatorname{in}}
\newcommand{\out}{\operatorname{out}}
\newcommand{\Dom}{\mathcal{X}}
\newcommand{\Rng}{\mathcal{Y}}
\newcommand{\PCM}{\operatorname{PCM}}
\newcommand{\CNOT}{\mathsf{CNOT}}
\newcommand{\Tg}{\mathsf{T}}
\newcommand{\Hg}{\mathsf{H}}
\newcommand{\Xg}{\mathsf{X}}
\newcommand{\Zg}{\mathsf{Z}}
\newcommand{\QECC}{\mathsf{QECC}}
% \newcommand{\Enc}{\mathsf{Enc}}
\newcommand{\enc}{\mathsf{enc}}
% \newcommand{\Dec}{\mathsf{Dec}}
\newcommand{\server}{\operatorname{S}}
\newcommand{\receiver}{\operatorname{R}}
\newcommand{\GL}{\operatorname{GL}}

% Sequential Authentication
\newcommand{\Auth}{{\sf Auth}}
\newcommand{\CAuth}{\mathsf{CAuth}} % Clifford auth
\newcommand{\TAuth}{\mathsf{TAuth}} % Trap auth
\newcommand{\KeyGen}{{\sf KeyGen}}
% \newcommand{\Enc}{{\sf Enc}}
% \newcommand{\Dec}{{\sf Dec}}
\newcommand{\relay}{{\sf Q}}
\newcommand{\flag}{{\sf F}}
\newcommand{\SApath}{\mathsf{path}}
\newcommand{\gray}[1]{\textcolor{gray}{#1}}
\newcommand{\AR}{\operatorname{AR}}

% Routing
\nc{\rCont}{\mathsf{continue}}
\nc{\rDrop}{\mathsf{drop}}
\nc{\rAbort}{\mathsf{abort}}




\newcommand{\ReEncCT}{\mathsf{ReEncC2T}}
\newcommand{\ReEncTC}{\mathsf{ReEncT2C}}
\newcommand{\circuit}{\mathsf{Qcircuit}}








