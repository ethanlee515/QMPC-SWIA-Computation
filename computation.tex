\section{Computation}

\nc{\vKeyGen}{\mathsf{KeyGen}}
\nc{\vEnc}{\mathsf{Enc}}
\nc{\vEval}{\mathsf{Eval}}
\nc{\verdec}{\mathsf{VerDec}}
\nc{\comp}{{\mathsf{comp}}}
\nc{\EvQFHE}{{\mathsf{EvQFHE}}}

For the sake of reduction, we'll first define an ideal functionality $\EvQFHE$ for \emph{extended} vQFHE which captures the core idea of our computation phase.
We'll prove its security. We then define our protocol $\comp$ for computation, and prove its security based on the security of $\EvQFHE$.

\subsection{Extended vQFHE}

\cite{magic_circuits} proposes the tuple $(\vKeyGen, \vEnc, \vEval, \verdec)$,
which implicitly defines a protocol between an always honest client and a potentially dishonest server.
We extend it by allowing interactions during computation and modifying the decoding process.

\begin{protocol}{The protocol $\EvQFHE$ for \emph{extended} verifiable QFHE}

	Public input:
	\begin{itemize}
		\item Sec. param $\lambda\in\bbN$
		\item Quantum circuit $C$ with input in $\cB^{\otimes n}$
	\end{itemize}

	Client's input:
	\begin{itemize}
		\item Some quantum state $\rho\in\cB^{\otimes n}$
	\end{itemize}

	Protocol:
	\begin{enumerate}
		\item The client generates $(sk, evk)\leftarrow\vKeyGen$ and $\sigma=\vEnc(sk, \rho)$. It sends $\sigma$ and $evk$ to the server.
		\item The server and client applies homomorphic computation. \Ethan{we'll make this precise later}
		\item The server sends the output ciphertext $\sigma'$ to the client.
		\item The client extracts the traps from $\sigma'$ and corrects the $\ket{+}$ to $\ket{0}$, obtaining $\tau\in\cB^{\otimes 2\lambda}$. (That is, on honest behavior, $\tau=\ket{0^{2\lambda}}$.)
			It samples $r, s\xleftarrow{\$}\zo^{2\lambda}$, then produces and sends $\tau'\leftarrow X^rZ^s\tau Z^sX^r$ to the server.
		\item The server measures $\tau'$ and sends the measurement results $r'$ back to the client.
		\item If $r=r'$, the client generates and outputs $\rho'$ by decoding the message portion of $\sigma'$. Otherwise the client outputs $\rho'\leftarrow\bot$.
	\end{enumerate}

\end{protocol}

In the simulator-based security of \cite{magic_circuits} there is no ``client";
instead, some implicit ``challenger" takes and decrypts the adversary's outputs, then sends only the results to the distinguisher.
Here we explicitly add the honest client to use the standard simulator-based security.

\begin{protocol}{The ideal box for $\EvQFHE$, $\trustp_\EvQFHE$}

	\begin{enumerate}
		\item The client sends $\rho$ to $\trustp_\EvQFHE$.
		\item The server sends $b\in\set{\acc, \rej}$ to $\trustp_\EvQFHE$.
		\item If $b=\acc$ then $\trustp_\EvQFHE$ sends $\rho'\leftarrow C(\rho)$ to the client, else it sends $\rho'\leftarrow\bot$ to the client.
	\end{enumerate}

\end{protocol}

I think a simulator $\cS_\EvQFHE$ is buildable.

\subsection{Computation}

Recall our candidate protocol is as below:

\begin{protocol}{Candidate protocol for input encoding}
	\begin{enumerate}
		\item QECC encoding
		\item AR everyone-to-server
		\item Re-Enc C-to-T
		\item QECC decoding
		\item Computation
		\item QECC encoding
		\item AR server-to-server
		\item Re-enc CT-to-C with traps-checking
		\item Re-enc C-to-T
		\item QECC decoding
	\end{enumerate}
\end{protocol}

Here I'll only capture steps $3$ to $8$ as below:

\begin{protocol}{Protocol $\comp$ for computation}

	\begin{enumerate}
		\item AR sender to server
		\item Re-Enc C to T
		\item Run some circuit
		\item AR server to server
		\item Re-Enc CT to T with checks
	\end{enumerate}

\end{protocol}

More precisely, the interface and the ideal functionality is as below:

\begin{protocol}{Ideal functionality $\trustp_\comp$ for computation}

	Public inputs:
	\begin{itemize}
		\item $\lambda\in\bbN$ security parameter
		\item $\vect{s_i}$, the size of $\Pc_i$'s inputs.
		\item Circuit family $C_K$ indexed by sets of dropped inputs.
	\end{itemize}

	Inputs:
	\begin{itemize}
		\item $\Pc_i$ has ensemble $\vect{\rho_{ij}}_{j=1}^{s_j}$
	\end{itemize}

	Protocol:
	\begin{enumerate}
		\item Everyone gives input to $\trustp_\comp$
		\item Server chooses $K$ for dropped inputs and sends it to $\trustp_\comp$
		\item $\trustp_\comp$ runs circuit on remaining inputs and Clifford encs the result
		\item Server gets enc'd outputs
		\item Adv chooses a set $K_2$ to keep. If the server is malicious, adversary can send $\bot$ to kill it instead.
		\item If server isn't killed, cMPC gets keys in $K_2$
	\end{enumerate}

	Outputs:
	\begin{itemize}
		\item Everyone gets $K_2$
		\item Server keeps enc'd outputs in $K_2$
		\item cMPC keeps keys in $K_2$
	\end{itemize}
\end{protocol}

\subsection{Reduction}

I'm still thinking how to go about this.

Originally I thought to assume for the sake of contradiction that there exists an attack against $\comp$,
and try to come up with an attack against $\EvQFHE$.
But maybe I can just directly build the simulator $\cS_\comp$ using $\cS_\EvQFHE$?
